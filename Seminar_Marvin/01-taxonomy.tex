%!TEX root = vorlage.tex
% Martin Thoma

\section{Taxonomy of Segmentation Algorithms}\label{sec:taxonomy}
The computer vision community has published a wide range of segmentation
algorithms so far. Those algorithms can be grouped by the kind of data they
operate on and the kind of segmentation they are able to produce.

The following subsections will give four different criteria by which
segmentation algorithms can be classified.

This survey describes fixed-class (see \cref{subsec:allowed-classes}),
single-class affiliation (see \cref{subsec:class-affiliation}) algorithms which
work on grayscale or colored single pixel images (see \cref{subsec:input-data})
in a completely automated, passive fashion (see \cref{subsec:operation-state}).

\subsection{Allowed classes}\label{subsec:allowed-classes}
Are the classes which are to be distinguished known at the time when the
algorithm is developed or trained, or might there occur several objects which
were never seen before?

Most algorithms work with a fixed set of classes; some even only work on binary
classes like \textit{foreground vs background}~\cite{4228537} or \textit{street
vs no street}~\cite{bittel2015pixel}.

There are also unsupervised segmentation algorithms which do not distinguish
classes at all (see \cref{subsec:unsupervised-traditional-segmentation}) as
well as segmentation algorithms which are able to recognize when they don't
know a class. For example, in~\cite{gould2008multi} a
\textit{void class} was added for classes which were not in the training set.


\subsection{Class affiliation of pixels}\label{subsec:class-affiliation}
Humans do an incredible job when looking at the world. For example, when we
see a glass of water standing on a table we can automatically say that there is
the glass and behind it the table, even if we only had a single image and were
not allowed to move. This means we assign the coordinates of the glass at the
same time two labels: Glass and table. Although there is much more work being
done on \textit{single class affiliation} segmentation algorithms, there is a
publication about \textit{multiple class affiliation}
segmentation~\cite{levin2008spectral}. Similarly, recent publications in
pixel-level object segmentation used layered models~\cite{yang2012layered}.


\subsection{Input Data}\label{subsec:input-data}
The available data which can be used for the inference of a segmentation varies
by application.

\begin{itemize}
    \item \textbf{Grayscale vs colored}: Grayscale images are commonly used in
          medical imaging such as \gls{MR} imaging or ultrasonography whereas
          colored photographs are obviously widespread.
    \item \textbf{Excluding or including depth data}: RGB-D, sometimes also
          called range~\cite{hoover1996experimental} is available in robotics,
          autonomous cars and recently also in consumer electronics such as
          Microsoft Kinect~\cite{6190806}.
    \item \textbf{Single image vs stereo images vs co-segmentation}: Single
          image segmentation is the most wide-spread kind of segmentation, but
          using stereo images was already tried in~\cite{boykov2001fast}. It
          can be seen as a more natural way of segmentation as most mammals
          have two eyes.
          Co-segmentation as in~\cite{1640859,collins2012random} is the problem
          of finding a consistent segmentation for multiple images. This problem
          can be seen in two ways: One the one hand, it can be seen as the problem
          of finding common objects in at least two images. On the other hand,
          every image after the first can be used as an additional source of
          information to find a meaningful segmentation. This idea can be
          extended to time series such as videos.
    \item \textbf{2D vs 3D}: Segmenting images is a 2D~segmentation task where
          the smallest unit is called a \textit{pixel}. In 3D data, such as
          volumetric X-ray CT images as they were used in~\cite{929615}, the
          smallest unit is called a voxel.
\end{itemize}


\subsection{Operation state}\label{subsec:operation-state}
The operation state of the classifying machine can either be \textit{active} as
in~\cite{schiebener2011segmentation,schiebener2012discovery} where robots can
move objects to find a segmentation or \textit{passive}, where the received
image cannot be influenced. However, being in a passive situation can be
divided further by separating segmentation algorithms working in a completely
\textit{automatic} fashion versus algorithms working in an \textit{interactive}
mode. One example would a system where the user clicks on the background or
marks a coarse segmentation and the algorithm finds a fine-grained
segmentation.
\cite{boykov2000interactive,rother2004grabcut,protiere2007interactive}~describe
systems which work in an interactive mode.
