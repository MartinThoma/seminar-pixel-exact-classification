%!TEX root = vorlage.tex

\section{Discussion}%
\label{sec:discussion}
Ohta et al. wrote~\cite{ohta1978analysis} 38~years ago. It is one of the first
papers mentioning semantic segmentation. In this time, a lot of work was done
and many different directions have been explored. Different kinds of semantic
segmentation have emerged.

This paper presents a taxonomy of those kinds of semantic segmentation and a
brief overview of completely automatic, passive, semantic segmentation
algorithms.

Future work includes a comparative study of those algorithms on publicly
available dataset such as the ones presented
in~\cref{table:segmentation-databases}. Another open question is the influence
of the problems described in~\cref{sec:problems}. This could be done using a
subset of the thousands of images of Wikipedia Commons, such as \href{https://commons.wikimedia.org/wiki/Category:Blurring}{https://commons.wikimedia.org/wiki/Category:Blurring} for blurred images.

A combination of different classifiers in an ensemble would be an interesting
option to explore in order to improve accuracy. Another direction which is
currently studied is combining classifiers such as neural networks with
\glspl{CRF}~\cite{zheng2015conditional}.
