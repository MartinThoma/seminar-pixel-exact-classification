%!TEX root = vorlage.tex
% Martin Thoma

\subsection{SVMs}\label{subsec:trad-SVM}%

% An Introduction to Support Vector Machines and Other Kernel-based Learning Methods
% Support-Vector Networks by Vapnik
% Learning with kernels: Support vector machines, regularization, optimization, and beyond
\Glspl{SVM} are well-studied binary classifiers which can be described by five
central ideas. For those ideas, the training data is represented as
$(\textbf{x}_i, y_i)$ where $\textbf{x}_i$ is the feature vector and $y_i \in
\Set{-1, 1}$ the binary label for training example $i \in \Set{1, \dots, m}$.


\begin{enumerate}
    \item If data is linearly seperable, it can be seperated by a hyperplane.
          There is one hyperplane which maximizes the distance to the
          datapoints. This hyperplane should be taken:\\
          \begin{equation*}
          \begin{aligned}
              \min_{\textbf{w}, b}\,&\frac{1}{2} \|\textbf{w}\|^2\\
              \text{subject to }& \forall_{i=1}^m y_i \cdot \underbrace{(\langle \textbf{w}, \textbf{x}_i\rangle + b)}_{\mathclap{\sgn \text{ applied to this gives the classification}}} \geq 0
          \end{aligned}
          \end{equation*}

          When the space gets normalized one can also demand a margin by re-formulating
          the constraint to
          $\forall_{i=1}^m y_i \cdot (\langle \textbf{w}, \textbf{x}_i\rangle + b) \geq 1$.
    \item The primal problem is to find the normal vector $\textbf{w}$ and the
          bias $b$. The dual problem is to express $\textbf{w}$ as a linear
          combination of the training data $\textbf{x}_i$:
          \[\textbf{w} = \sum_{i=1}^m \alpha_i y_i \textbf{x}_i\]
          where $y_i \in \Set{-1, 1}$ represents the class of the training
          example.
    \item Not every dataset is linearly seperable. This problem is approached
          by transforming the dataset with a non-linear mapping into a higher
          dimensional space $\Phi$.
    \item Implicitly using a high-dimensional (probably $\infty$-dimensional)
          space. % TODO: http://www.jstor.org/stable/pdf/25464664.pdf?acceptTC=true
          % Schölkopf: Learning with kernels
          % Kernel-Trick
    \item Introduction of slack variables to relax the requirement of linear
          seperability. The trade-off between accepting some errors and a more
          complex model is weighted by a parameter $C \in \mathbb{R}_0^+$. The
          bigger $C$, the more errors are accepted. The new optimization
          problem is:
          \begin{equation*}
          \begin{aligned}
              \min_{\textbf{w}}\,&\frac{1}{2} \|\textbf{w}\|^2 + C \cdot \sum_{i} \xi_i\\
              \text{subject to }& \forall_{i=1}^m y_i \cdot (\langle \textbf{w}, \textbf{x}_i\rangle + b) \geq 1 - \xi_i
          \end{aligned}
          \end{equation*}
\end{enumerate}

The described \glspl{SVM} can only distinguish between two classes. Common
strategies to expand those binary classifiers to multi-class classification is
the \textit{one-vs-all} and the \textit{one-vs-one} strategy. In the one-vs-all
strategy $n$ classifiers have to be trained which can distingish one of the $n$
classes against all other classes. In the one-vs-one strategy $\frac{n^2 - n}{2}$
classifiers are trained; one classifier for each pair of classes.

A detailed description of \glspl{SVM} can be found in \cite{burges1998tutorial}.

\Glspl{SVM} are used by \cite{yang2012layered} on the 2009 and 2010 PASCAL
segmentation challenge~\cite{everingham2010pascal}. They did not hand their
classifier in to the challenge itself, but calculated an average rank~of~7
amongst the different categories.

\cite{felzenszwalb2010object} also used an SVM based method and achieved the
7th~rank in the 2010 PASCAL segmentation callenge by mean accuracy.