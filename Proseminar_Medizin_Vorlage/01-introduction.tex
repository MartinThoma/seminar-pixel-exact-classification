%!TEX root = vorlage.tex
% Marvin Teichmann and Martin Thoma
\section{Introduction}\label{sec:introduction}
Semantic segmentation is the task of clustering parts of images together which
belong to the same object class. This type of algorithm has several use-cases
such as detecting road % detecting objects in robotics (TODO: cite),
signs~\cite{4220659}, detecting tumors in medicine~\cite{moon2002automatic},
detecting medical instruments in operations~\cite{wei1997automatic}, colon
crypts segmentation~\cite{cohen2015memory}, land use and land cover
classification~\cite{huang2002assessment}. In contrast, non-semantic
segmentation only clusters pixels together based on general characteristics of
single objects. Hence the task of non-semantic segmentation is not
well-defined, as many different segmentations might be acceptable.

Several applications of segmentation are listed
in~\cite{annurev.bioeng.2.1.315}.

Object detection, in comparison to semantic segmentation, has to distinguish
different instances of the same object. While having a semantic segmentation is
certainly a big advantage when trying to get object instances, there are a
couple of problems: neighboring pixels of the same class might belong to
different object instances and regions which are not connected my belong to the
same object instace. For example, a tree in front of a car which visually
divides the car into two parts. Some papers report to have used
both~\cite{tighe2014scene}.

% Pixelwise segmentation is an imporant sub-task of many applications and can
% be applied to support even more applications: Understanding images (TODO:
% Cite paper google automatically labeling images), object detection (TODO:
% Cite a paper probably by Asfour), street detection (TODO: Cite), face
% detection (TODO: cite) and medical instrument detection (TODO: cite) are only
% a couple of the possible applications.

% In many applications speed and accuracy is crucial. In this paper, we
% summarize the current state of semantic segmentation.

This paper is organized as follows: We begin by giving a taxonomy of
segmentation algorithms. %in \cref{sec:taxonomy}.
A summary of quality measures and datasets which are used for semantic
segmentation. A summary of the most important traditional segmentation
algorithm and their characteristics follows, as well as a similar summary of
recently published semantic segmentation algorithms which are based on neural
networks. The next section informes the reader about typical problematic cases
for segmentation algorithms. As speed is one of the hard requirements in
practical applications, we outline some of the actions that can be applied to
improve the speed of segmentation algorithms. Finally, we mention some of the
publications which examine automatic image enhancement.
