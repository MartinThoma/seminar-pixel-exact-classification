%!TEX root = vorlage.tex
% Martin Thoma
\subsection{Quality measures for evaluation}\label{subsec:quality-measures}

A performance measure is a crucial part of any machine learning system, but
there are other measures of quality which matter when segmentation algorithms
are compared. This section gives an overview of those quality measures.


\subsubsection{Accuracy}
One way to compare segmentation algorithms is by the accuracy of the
segmentation they find.

One accuracy measure is \textit{Normalized Probabilistic Rand} (NPR) index
which was introduced in \cite{unnikrishnan2005measure}.

Another measure was introduced by~\cite{celebi2009improved}. TODO: Explain it

Another framework is introduced by~\cite{jaber2010probabilistic}. TODO: Explain
it

\subsubsection{Speed}\label{subsubsec:speed-quality-measure}
One obvious quality measure is speed. For many applications, this is a hard
requirement. For example, in the case of self-driving cars an algorithm which
classifies pixel as street or no-street and thus makes a semantic segmentation,
every image needs to be processed within
\SI{20}{\milli\second}~\cite{bittel2015pixel}.

TODO: Which categories are there?

TODO: Health informatics: Which times?


\subsubsection{Stability}\label{subsubsec:stability-quality-measure}
The stability of segmentation is a desirable quality measure. One the one hand,
there are some variants of images like slight bluring, gaussian noise and
similar which should not change the segmentation at all. Also, two images which
show a slight change in perspective should also only result in slight changes
in the segmentation.

The other desirable stability criterium is parameter choice. If the
segmentation algorithm has hyperparameters, then slightly changing those should
also only result in minor differences of the resulting segmentations.

(TODO: Explain why)
TODO: Cite who uses it.
