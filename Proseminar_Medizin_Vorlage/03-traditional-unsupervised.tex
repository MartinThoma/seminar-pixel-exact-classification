%!TEX root = vorlage.tex
% Martin Thoma

\subsection{Unsupervised Segmentation}%
\label{subsec:unsupervised-traditional-segmentation}%

Unsupervised segmentation algorithms can be used in supervised segmentation as
another source of information or to refine a segmentation. While unsupervised
segmentation algorithms can never be semantic, they are well-studied and
deserve at least a very brief overview.

Semantic segmentation algorithms store information about the classes they were
trained to segment while non-semantic segmentation algorithms try to detect
consistent regions and region boundaries.

\subsubsection{Clustering Algorithms}
The $k$-means algorithm is a general-purpose clustering algorithm which
requires the number of clusters to be given beforehand. Initially, it places
the $k$ centroids randomly in the feature space. Then it assigns each
data point to the nearest centroid, moves the centroid to the center of the
cluster and continues the process until a stopping criterion is reached. A
faster variant is described in \cite{hartigan1975clustering}.

$k$-means was applied by~\cite{chen1998image} for medical image segmentation.

Another clustering algorithm is the mean-shift algorithm which was introduced
by~\cite{comaniciu2002mean} for segmentation tasks. The algorithm finds the
cluster centers by initializing centroids at random seed points and iteratively
shifting them to the mean coordinate within a certain range. Instead of taking
a hard range constraint, the mean can also be calculated by using any kernel.
This effectively applies a weight to the coordinates of the points. The mean
shift algorithm will give cluster centers at positions with a highest local
density of points.


\subsubsection{Graph Based Image Segmentation}%
\label{subsec:graph-based-image-segmentation}%
Graph-based image segmentation algorithms typically interpret pixels as
vertices and an edge weight is a measure of dissimilarity such as the
difference in color~\cite{felzenszwalb2004efficient,FelzenszwalbGraphCode}.
There are several different candidates for edges. The 4-neighborhood (north,
east, south west) or an 8-neighborhood (north, north-east, east, south-east,
south, south-west, west, north-west) are plausible choices.
One way to do the cuts is by building a minimum spanning tree and removing
edges above a threshold. This threshold can either be constant, adapted to the
graph or adjusted by the user. After the edge-cutting step, the connected
components are the segments.

A graph-based method which got the \nth{2} rank in the Pascal VOC 2010
challenge~\cite{everingham2010pascal} is described
in~\cite{carreira2010constrained}. The system makes heavy use of the multi-cue
contour detector globalPb~\cite{4587420} and needs about \SI{10}{\giga\byte}
of main memory~\cite{Carreira2011}.


\subsubsection{Random Walks}

Random walks belong to the graph-based image segmentation algorithms. Random
walk image segmentation usually works as follows: Some seed points are placed
on the image for the different objects in the image. From every single pixel,
the probability to reach the different seed points by a random walk is
calculated. This is done by taking image gradients as described in
\cref{subsec:features} for \gls{HOG} features. The class of the pixel is the
class of which a seed point will be reached with highest probability. At first,
this is an interactive segmentation method, but it can be extended to be
non-interactive by using another segmentation methods output as seed points.

It was shown in~\cite{meilpa2001learning} that normalized cuts
(NCuts)~\cite{shi2000normalized} can be expressed with random walks.


\subsubsection{Active Contour Models}

\Glspl{ACM} are algorithms which segment images roughly along edges, but try
also try to find a border which is smooth. This is done by defining a so called
\textit{energy function} which will get minimized. They were initially
described in~\cite{kass1988snakes}. \Glspl{ACM} can be used to segment an image
or to refine segmentation as it was done in~\cite{atkins1998fully} for brain
\gls{MR} images.


\subsubsection{Watershed Segmentation}\label{subsec:watershed}
The watershed algorithm takes a grayscale image and interprets it as a height
map. Low values are catchment basins and the higher values between two
neighboring catchment basins is the watershed. The catchment basins should
contain what the developer wants to capture. This implies that those areas
must be dark on grayscale images. The algorithm starts to fill the basins from
the lowest point. When two basins get connected, a watershed is found. The
algorithm stops when the highest point is reached.

A detailed description of the watershed segmentation algorithm is given
in~\cite{roerdink2000watershed}.

The watershed segmentation was used in~\cite{1260033} to segment white blood
cells. As the authors describe, the segmentation by watershed transform has
two flaws: Over-segmentation due to local minima and thick watersheds due to
plateaus.