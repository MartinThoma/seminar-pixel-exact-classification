%!TEX root = vorlage.tex
% Marvin Teichmann and Martin Thoma
\section{Taxonomy of Segmentation Algorithms}\label{sec:taxonomy}
As far as the authors of this paper know, a complete taxonomy of segmentation
algorithms was never done. While reading publications about segmentation
algorithms, the authors of this paper noticed the following ways to describe
segmentation algorithms.

\textbf{By possible classes}: Are the classes which should be distinguised
known at the time when the algorithm is developed / trained or might there
occur several objects which were never seen before?
\begin{itemize}
    \item Fixed-class: All classes are known at training time.
          \begin{itemize}
              \item Binary: street/no street.
          \end{itemize}
    \item Open-class: There might be completely new classes
\end{itemize}

\textbf{By class affiliation of pixels}: Humans do an increadible job when
looking at the world. For example, when we see a glass of water standing on a
table we can automatically say that there is the glass and behind it the table,
even if we only had a single image and were not allowed to move. This means we
assign the coordinates of the glass at the same time two labels: Glass and
table. Although there is much more work being done on \textit{single class
affiliation} segmentation algorithms, there is a publication about multiple
class affiliation segmentation~\cite{levin2008spectral}. Similarly, recent
publications in pixel-level object segmentation used layered
models~\cite{yang2012layered}.

\textbf{By input data}:
\begin{itemize}
    \item greyscale / colored / with depth (RGB-D), sometimes also called
          range~\cite{hoover1996experimental}
    \item single image / stereo images as in \cite{boykov2001fast} /
          time~series / co-segmentation as in~\cite{1640859,collins2012random}
    \item pixels / voxels \cite{wolz2012multi}
\end{itemize}

\textbf{Operation state}: The operation state of the classifying machine can
either be \textit{active} as in
\cite{schiebener2011segmentation,schiebener2012discovery} or passive, where the
received image cannot be influenced. However, being in a passive situation can
be divided further by separating segmentation algorithms working in a
completely automatic fashing versus algorithms working in an interactive mode.
One example would a system where the user clicks on the background or marks a
coarse segmentation and the algorithm finds a fine-grained segmentation.
\cite{boykov2000interactive,rother2004grabcut,protiere2007interactive}~describe
systems which work in an interactive mode.

This survey describes fixed-class, single-class affiliation algorithms which
work on greyscale or colored single pixel images in a completely automated,
passive fashion.
