%!TEX root = vorlage.tex
% Martin Thoma

\subsection{Markov Random Fields}\label{subsec:markov-random-fields}
% TODO: Probability space?
A \Gls{MRF} is a tupel $(G, X)$, where $G=(V,E)$ is an undirected graph and
$X=(X_V)_{v \in V}$ is a set of random variables which satisfy the pairwise
Markov property, the local Markov property and the global Markov property:

% https://en.wikipedia.org/wiki/Markov_random_field#Definition
% \begin{itemize}
%     \item \textbf{Pairwise Markov property}: $X_u \independent X_v | X_{V \setminus \Set{u, v}}$ if $\Set{u,v} \notin E$
%     \item \textbf{Local Markov property}: $X_u \independent X_{V \setminus \Set{v | v \text{is } u \text{ or a neighbor of } u}} | X_{\text{neighbors of } v}$
%     \item \textbf{Global Markov property}: $X_A \independent X_B | X_S$ where
%           every path from a node in $A$ to a node in $B$ passes through $S$
% \end{itemize}

% is a probabilisitic model. They have the Markov property which is
% typically displayed in an undirected graph which shows dependencies between the
% random variables.

\Glspl{CRF} and Boltzmann Machines are a variations of Markov random fields.

% http://www.mathunion.org/ICM/ICM1986.2/Main/icm1986.2.1496.1517.ocr.pdf
%
% Segmentation of brain MR images through a hidden Markov random field model
% and the expectation-maximization algorithm \cite{zhang2001segmentation}

% In short: specify locally and model globally.

% define only local properties. allows by transitivity to get a model
% for global properties.

% definition:

% A \gls{MRF} is a countable set of random variables. In the task of segmentation,
% they are often index-based spacial positions.

% \begin{itemize}
%     \item What is your set of random models? (e.g. Ising model: one for each pixel)
%           \begin{itemize}
%               \item Irving-Model: \cite{boykov2000interactive}
%               \item Potts-Model: \cite{boykov2001fast}
%           \end{itemize}
% \end{itemize}

% On those, define an undirected hypergraph $G(V, E)$. $V$ are pixels, $E$ is
% typically defined by neighboring pixels.

% good: 

% \begin{itemize}
%     \item Convenient modeling:Just write down energie function (in some cases)
%           do define entire model
%     \item fast inference (only 1 or 2 variants)
%     \item sometimes good performance
% \end{itemize}

% bad:

% \begin{itemize}
%     \item learning is difficult
% \end{itemize}



% ---


From \cite{yang2012layered}:

> In contrast, semantic segmentation models have largely been built on top of
Markov Random Field (MRF) models which enforce smoothness across pixel labels

\begin{itemize}
    \item X. He, R. Zemel, and M. Carreira-Perpinan, “Multiscale Conditional
          Random Fields for Image Labeling,” Proc. IEEE CS Conf. Computer
          Vision and Pattern Recognition, vol. 2, 2004.
    \item A. Torralba, K. Murphy, and W. Freeman, “Contextual Models for
          Object Detection Using Boosted Random Fields,” Proc. Advances in
          Neural Information Processing Systems, 2004.
    \item S. Kumar and M. Hebert, “A Hierarchical Field Framework for
          Unified Context-Based Classification,” Proc. 10th IEEE Int’l Conf.
          Computer Vision, vol. 2, 2005.
    \item \Glspl{CRF} were applied in \cite{shotton2006textonboost}.
    \item Z. Tu, “Auto-Context and Its Application to High-Level Vision
          Tasks,” Proc. IEEE Conf. Computer Vision and Pattern Recognition,
          2008.
\end{itemize}


A method similar to \glspl{CRF} was proposed in~\cite{gonfaus2010harmony}.
The system of Gonfaus~et.al. ranked~1st by mean accuracy in the segmentation
task of the PASCAL VOC 2010 challenge~\cite{everingham2010pascal}.

TODO:

\begin{itemize}
    \item "Associative hierarchical CRF" by "Oxford Brookes University", "Lubor Ladicky Christopher Russell Philip Torr", see \href{http://host.robots.ox.ac.uk/pascal/VOC/voc2010/results/index.html}{pascal voc 2010}
\end{itemize}