%!TEX root = vorlage.tex

\subsection{Post-processing methods}\label{subsec:post-processing-methods}%
Post-processing methods are an important part of traditional pixel-level
segmentation approaches. They refine a found segmentation and remove obvious
errors. For example, the morphological operations \textit{opening} and
\textit{closing} can remove noise. The opening operation is a dilation followed
by a erosion. This removes tiny segments. The closing operation is a erosion
followed by a dilation. This removes tiny gaps in otherwise filled regions.
They were used in~\cite{chen1998image} for biomedical image segmentation.

Another way of refinement of the found segmentation is by adjusting the
segmentation to match close edges. This was used in~\cite{brox2011object} with
an ultra-metric contour map~\cite{arbelaez2009contours}.

Active contour models are another example of a post-processing
method~\cite{kass1988snakes}.
