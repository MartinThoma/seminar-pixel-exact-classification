%!TEX root = vorlage.tex

\section{Neural Networks for Semantic Segmentation}\label{sec:nn}

Artificial neural networks are classifiers which are inspired by biologic
neurons. Every single artificial neuron has some inputs which get weighted and
sumed up. Then, the neuron applies a so called \textit{activation function} to
the weighted sum and gives an output. Those neurons can take either a feature
vector as input or the output of other neurons. In this way, they build up
feature hierarchies.

The parameters they learn are the \textit{weights} $w \in \mathbb{R}$. They get
learned by gradient descent. To do so, an error function --- usually
cross-entropy or mean squared error --- is necessary. For the gradient descent
algorithm, one sees the labeled training data as given, the weights as
variables and the error function as a surface in this weight-space. Minimizing
the error function in the weight space adapts the neural network to the
problem.

There are lots of ideas around neural networks like regularization, better
optimization algorithms, automatically building up architectures, design
choices for activation functions. I will not go into detail here, but I want to
outline some of the mayor breakthroughs.

\Glspl{CNN} are neural networks which learn image filters. They drastically
reduce the number of parameters which have to be learned while being still
general enough for the problem domain of images. This was shown by Alex
Krizhevsky~et~al. in~\cite{krizhevsky2012imagenet}. One major idea was a clever
regularization called \textit{dropout training}, which set the output of
neurons while training randomly to zero. Another contribution was the usage of
an activation function called \textit{rectified linear unit}:
\[\varphi_{\text{ReLU}}(x) = \max(0, x)\]
Those are much faster to train than the commonly used sigmoid activation
functions
\[\varphi_{\text{Sigmoid}}(x) = \frac{1}{e^{-x} + 1}\]
Krizhevsky~et~al. implemented those ideas and participated in the
\gls{ILSVRC}. The best other system, which used SIFT features and Fisher
Vectors, had a performance of about \SI{25.7}{\percent} while the network by
Alex Krizhevsky~et~al. got \SI{17.0}{\percent} error rate on the ILSVRC-2010
dataset. As a preprocessing step, they downsampled all images to a fixed size
of $\SI{256}{\pixel} \times \SI{256}{\pixel}$ before they fed the features into
their network. This network is commonly known as \textit{AlexNet}.

Since AlexNet was developed, a lot of different neural networks have been
proposed. One interesting example is~\cite{pinheiro2013recurrent}, where
a recurrent \gls{CNN} for semantic segmentation is presented.

Another notable paper is~\cite{long2014fully}. The algorithm presented there
makes use of a classifying network such as AlexNet, but applies the complete
network as an image filter. This way, each pixel gets a probability
distribution for each of the trained classes. By taking the most likely class,
a semantic segmentation can be done with arbitrary image sizes.

A very recent publication by Dai~et~al.~\cite{dai2015instance} showed that
segmentation with much deeper networks is possible and achieves better results.

More detailed explanations to neural networks can be found
in~\cite{Teichmann2016}.
