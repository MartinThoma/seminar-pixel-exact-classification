%!TEX root = vorlage.tex

\subsection{Markov Random Fields}\label{subsec:markov-random-fields}
\Glspl{MRF} are undirected probabilistic graphical models which are wide-spread
model in computer vision. Detailed introductions to \glspl{MRF} are given by
\cite{blake2011markov,murphy2012machine}. The following will only give a rough
idea of \glspl{MRF}.

Let $G=(\mathcal{V}, \mathcal{E})$ be the associated undirected graph and
$\mathcal{C}$ be the set of all maximal cliques in that graph. Nodes represent
random variables $\mathbf{x}$ and edges represent conditional dependencies.
Just like in \cref{subsec:graph-based-image-segmentation}, the
4-neighborhood~\cite{shotton2006textonboost} and the 8-neighborhood are
reasonable choices for constructing the graph.

Now the probability of $\mathbf{x}$ can be expressed as
\[P(\mathbf{x}) = \frac{1}{Z} e^{-E(\mathbf{x})}\]
where $Z = \sum_{\mathbf{x}} e^{-E(\mathbf{x})}$ is a normalization term called
the \textit{partition function} and $E$ is called the \textit{energy function}.

According to~\cite{murphy2012machine}, the most common way of inference over
the posterior \gls{MRF} in computer vision problems is \gls{MAP} estimation.




\glspl{MRF} are used by \cite{zhang2001segmentation} and \cite{moser2012markov}
for image segmentation.

% Characterizations of MRF:
% Label space: binary vs multi-label; homogeneous vs heterogeneous
% Order: unary vs pairwise vs higher-order
% Structure: chain vs tree vs grid vs general graph; neighborhood size
% Potentials: submodular, convex, compressible

% Markov Random Fields (A Rough Guide)
% http://www.mee.tcd.ie/~sigmedia/pmwiki/uploads/Main.Tutorials/mrf_tutorial.pdf

% Markov Random Fields and Stochastic Image Models
% http://cis-linux1.temple.edu/~latecki/Courses/RobotFall08/Papers/MRFBauman.pdf

% Markov Random Fields
% http://signal.ee.psu.edu/mrf.pdf

% Markov Random Fields for Computer Vision (Part 1)
% http://users.cecs.anu.edu.au/~sgould/papers/part1-MLSS-2011.pdf  --- very nice!

% Markov Random Fields and Stochastic Image Models
% https://engineering.purdue.edu/~bouman/publications/tutorials/mrf_tutorial/view.pdf

% Markov Random Fields in Image Segmentation
% http://web.inf.u-szeged.hu/~ssip/2011/PDF/ssip2011_Kato.pdf

% Book about markov random fields in computer vision:
% https://mitpress.mit.edu/sites/default/files/titles/content/9780262015776_sch_0001.pdf


% \Glspl{CRF} and Boltzmann Machines are a variations of Markov random fields.

% Markov Random Field Image Models and Their Applications to Computer Vision
% http://www.mathunion.org/ICM/ICM1986.2/Main/icm1986.2.1496.1517.ocr.pdf


\subsection{Conditional Random Fields}\label{subsec:conditional-random-fields}

\Glspl{CRF} are \glspl{MRF} where all clique potentials are conditioned on
input features~\cite{murphy2012machine}. This means, instead of learning the
distribution $P(\mathbf{y}, \mathbf{x})$, the task is reformulated to learn the
distribution $P(\mathbf{y}| \mathbf{x})$. One consequence of this reformulation
is that \glspl{CRF} need much less parameters as the distribution of
$\mathbf{x}$ does not have to be estimated. Another advantage of \glspl{CRF}
compared to \glspl{MRF} is that no distribution assumption about $\mathbf{x}$
has to be made.

A \gls{CRF} has the partition function $Z$:
\[Z(\mathbf{x}) = \sum_{\mathbf{y}} P(\mathbf{x}, \mathbf{y})\]

and joint probability distribution

\[P(\mathbf{y} | \mathbf{x}) = \frac{1}{Z(\mathbf{x})} \prod_{c \in \mathcal{C}} \psi_c(\mathbf{y}_c | \mathbf{x})\]

where $\psi_c$ are so called \textit{clique potentials}. The simplest way to
define them would be the count of the class $\mathbf{y}_c$ given $\mathbf{x}$
added with a positive smoothing constant to prevent the complete term from
getting zero.

Typically, random variables $\mathbf{y}$ represent the class of a single pixel,
random variables $\mathbf{x}$ represent a pixel values and edges represent
pixel neighborhood in computer vision problems segmentation problems where
\glspl{MRF} are used. Accordingly, the random variables $\mathbf{y}$ live on
$1, \dots, \text{nr of classes}$ and the random variables $\mathbf{x}$
typically live on $0, \dots, 255$ or $[0, 1]$.

\Glspl{CRF} as described in~\cite{associative09} have reached top performance
in PASCAL VOC 2010~\cite{VOC2010Results} and are also used in
\cite{multiscale04,shotton2006textonboost} for semantic segmentation.

A method similar to \glspl{CRF} was proposed in~\cite{gonfaus2010harmony}.
The system of Gonfaus~et.al. ranked~\nth{1} by mean accuracy in the segmentation
task of the PASCAL VOC 2010 challenge~\cite{everingham2010pascal}.

An introduction to \glspl{CRF} is given by~\cite{sutton2011introduction}.
