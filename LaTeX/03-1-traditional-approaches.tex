%!TEX root = vorlage.tex

\section{Traditional Approaches}\label{sec:traditional-approaches}%
Image segmentation algorithms which use traditional approaches, hence don't
apply neural networks and make heavy use of domain knowledge, are wide-spread
in the computer vision community. Features which can be used for segmentation
are described in \cref{subsec:features}, a very brief overview of unsupervised,
non-semantic segmentation is given in
\cref{subsec:unsupervised-traditional-segmentation}, Random Decision
Forests are described in \cref{subsec:random-forests}, Markov Random Fields in \cref{subsec:markov-random-fields}
 and \glspl{SVM} in
\cref{subsec:trad-SVM}.
Postprocessing is covered in \cref{subsec:post-processing-methods}.

It should be noted that algorithms can use combination of methods. For example,
\cite{tighe2014scene} makes use of a combination of a \gls{SVM} and a
\gls{MRF}. Also, auto-encoders can be used to learn features which in turn
can be used by any classifier.

% TODO: The respective advantages of the classifiers are discussed in
% \cref{subsec:traditional-approaches-discussion}.

%!TEX root = vorlage.tex

\subsection{Features and Preprocessing methods}\label{subsec:features}%
The choice of features is very important in traditional approaches.
The most commonly used features are explained in the following.

\subsubsection{Pixel Color}
Pixel color in different image spaces (e.g. 3~features for RGB, 3~features for
HSV, 1~feature for the gray-value) are the most widely used features. A typical
image is in the RGB color space, but depending on the classifier and the
problem another color space might result in better segmentations. RGB, YcBcr,
HSL, Lab and YIQ are some examples used by \cite{cohen2015memory}. No single
color space has been proven to be superior to all others in all
contexts~\cite{cheng2001color}. However, the most common choices seem to be RGB
and HSI. Reasons for choosing RGB is simplicity and the support by programming
languages, whereas the choice of the HSI color space might make it simpler for
the classifier to get invariant to illumination. One reason for choosing
CIE-L*a*b* color space is that it approximates human perception of
brightness~\cite{kasson1992analysis}. It follows that choosing the L*a*b color
space helps algorithms to detect structures which are seen by humans. Another
way of improving the structure within an image is histogram equalization, which
can be applied to improve contrast~\cite{pizer1987adaptive,4228537}.

\subsubsection{Histogram of oriented Gradients}
\Gls{HOG} features interpret the image as a discrete function
$I: \mathbb{N}^2 \rightarrow \Set{0, \dots, 255}$ which maps the position $(x,
y)$ to a color. For each pixel, there are two gradients: The partial derivative
of $x$ and $y$. Now the original image got transformed to two feature maps of
equal size which represents the gradient. These feature maps are splitted into
patches and a histogram of the directions is calculated for each patch.
\gls{HOG} features were proposed in~\cite{1467360} and are used
in~\cite{bourdev2010detecting,felzenszwalb2010object} for segmentation tasks.


\subsubsection{SIFT}
\Gls{SIFT} feature descriptors find keypoints in an image. The image patch of
the size $16 \times 16$ around the keypoint is taken. This patch is divided in
$16$ distinct parts of the size $4 \times 4$. For each of those parts a
histogram of 8~orientations is calculated similar as for \gls{HOG} features.
This results in a 128-dimensional feature vector for each keypoint.

\Gls{SIFT} is described in detail in~\cite{raey}.


\subsubsection{BOV}
\Gls{BOV}, also called \textit{bag of keypoints}, is based on vector
quantization. Similar to \gls{HOG} features, \gls{BOV} features are histograms
which count the number of occurrences of certain patterns within a patch of the
image. \Gls{BOV} are described in~\cite{csurka2004visual} and used in
combination with \gls{SIFT} feature descriptors in~\cite{csurka2008simple}.


\subsubsection{Poselets}
\textit{Poselets} rely on manually added extra keypoints such as \enquote{right
shoulder}, \enquote{left shoulder}, \enquote{right knee} and \enquote{left
knee}. They were originally used for human pose estimation. Finding those extra
keypoints is easily possible for well-known image classes like humans. However,
it is difficult for classes like airplanes, ships, organs or cells where the
human annotators do not know the keypoints. Additionally, the keypoints have to
be chosen for every single class. There are strategies to deal with those
problems like viewpoint-dependent keypoints. Poselets were used
in~\cite{bourdev2010detecting} to detect people and in~\cite{brox2011object}
for general object detection of the PASCAL VOC dataset.

\subsubsection{Textons}\label{subsubsec:textons}
A \textit{texton} is the minimal building block of vision. The computer vision
literature does not give a strict definition for textons, but edge detectors
could be one example. One might argue that deep learning techniques with
\glspl{CNN} learn textons in the first filters.

An excellent explanation of textons can be found in~\cite{zhu2005textons}.


\subsubsection{Dimensionality Reduction}
High-resolution images have a lot of pixels. Having one or more feature per
pixel results in many features. This makes training difficult while the higher
resolution might not contain much more information. A simple approach to deal
with this is downsampling the high-resolution image to a low-resolution
variant. Another way of doing dimensionality reduction is \gls{PCA}, which is
applied by~\cite{chen2011pixel}. The idea behind \gls{PCA} is to find a
hyperplane on which all feature vectors can be projected with a minimal loss of
information. A detailed description of \gls{PCA} is given
by~\cite{smith2002tutorial}.

One problem of \gls{PCA} is the fact that it does not distinguish different
classes. This means it can happen that a perfectly linearly separable set of
feature vectors gets not separable at all after applying \gls{PCA}.

There are many other techniques for dimensionality reduction. An overview and
a comparison over some of them is given by~\cite{van2009dimensionality}.

%!TEX root = vorlage.tex

\subsection{Unsupervised Segmentation}%
\label{subsec:unsupervised-traditional-segmentation}%

Unsupervised segmentation algorithms can be used in supervised segmentation as
another source of information or to refine a segmentation. While unsupervised
segmentation algorithms can never be semantic, they are well-studied and
deserve at least a very brief overview.

Semantic segmentation algorithms store information about the classes they were
trained to segment while non-semantic segmentation algorithms try to detect
consistent regions or region boundaries.


\subsubsection{Clustering Algorithms}
Clustering algorithms can directly be applied on the pixels, when one gives
a feature vector per pixel. Two clustering algorithms are $k$-means and the
mean-shift algorithm.

The $k$-means algorithm is a general-purpose clustering algorithm which
requires the number of clusters to be given beforehand. Initially, it places
the $k$ centroids randomly in the feature space. Then it assigns each
data point to the nearest centroid, moves the centroid to the center of the
cluster and continues the process until a stopping criterion is reached. A
faster variant is described in \cite{hartigan1975clustering}.

$k$-means was applied by~\cite{chen1998image} for medical image segmentation.

Another clustering algorithm is the mean-shift algorithm which was introduced
by~\cite{comaniciu2002mean} for segmentation tasks. The algorithm finds the
cluster centers by initializing centroids at random seed points and iteratively
shifting them to the mean coordinate within a certain range. Instead of taking
a hard range constraint, the mean can also be calculated by using any kernel.
This effectively applies a weight to the coordinates of the points. The mean
shift algorithm finds cluster centers at positions with a highest local
density of points.


\subsubsection{Graph Based Image Segmentation}%
\label{subsec:graph-based-image-segmentation}%
Graph-based image segmentation algorithms typically interpret pixels as
vertices and an edge weight is a measure of dissimilarity such as the
difference in color~\cite{felzenszwalb2004efficient,FelzenszwalbGraphCode}.
There are several different candidates for edges. The 4-neighborhood (north,
east, south west) or an 8-neighborhood (north, north-east, east, south-east,
south, south-west, west, north-west) are plausible choices.
One way to do the cuts is by building a minimum spanning tree and removing
edges above a threshold. This threshold can either be constant, adapted to the
graph or adjusted by the user. After the edge-cutting step, the connected
components are the segments.

A graph-based method which got the \nth{2} rank in the Pascal VOC 2010
challenge~\cite{everingham2010pascal} is described
in~\cite{carreira2010constrained}. The system makes heavy use of the multi-cue
contour detector globalPb~\cite{4587420} and needs about \SI{10}{\giga\byte}
of main memory~\cite{Carreira2011}.


\subsubsection{Random Walks}

Random walks belong to the graph-based image segmentation algorithms. Random
walk image segmentation usually works as follows: Some seed points are placed
on the image for the different objects in the image. From every single pixel,
the probability to reach the different seed points by a random walk is
calculated. This is done by taking image gradients as described in
\cref{subsec:features} for \gls{HOG} features. The class of the pixel is the
class of which a seed point will be reached with highest probability. At first,
this is an interactive segmentation method, but it can be extended to be
non-interactive by using another segmentation methods output as seed points.

It was shown in~\cite{meilpa2001learning} that normalized cuts
(NCuts)~\cite{shi2000normalized} can be expressed with random walks.


\subsubsection{Active Contour Models}

\Glspl{ACM} are algorithms which segment images roughly along edges, but try
also try to find a border which is smooth. This is done by defining a so called
\textit{energy function} which will get minimized. They were initially
described in~\cite{kass1988snakes}. \Glspl{ACM} can be used to segment an image
or to refine segmentation as it was done in~\cite{atkins1998fully} for brain
\gls{MR} images.


\subsubsection{Watershed Segmentation}\label{subsec:watershed}
The watershed algorithm takes a grayscale image and interprets it as a height
map. Low values are catchment basins and the higher values between two
neighboring catchment basins is the watershed. The catchment basins should
contain what the developer wants to capture. This implies that those areas
must be dark on grayscale images. The algorithm starts to fill the basins from
the lowest point. When two basins get connected, a watershed is found. The
algorithm stops when the highest point is reached.

A detailed description of the watershed segmentation algorithm is given
in~\cite{roerdink2000watershed}.

The watershed segmentation was used in~\cite{1260033} to segment white blood
cells. As the authors describe, the segmentation by watershed transform has
two flaws: Over-segmentation due to local minima and thick watersheds due to
plateaus.

%!TEX root = vorlage.tex

\subsection{Random Decision Forests}\label{subsec:random-forests}

Random Decision Forests were first proposed in~\cite{ho1995random}. This type
of classifier applies techniques called \textit{ensemble learning}, where
multiple classifiers get trained and a combination of their hypotheses is
used. One ensemble learning technique is the \textit{random subspaces} method
where each classifier gets trained on a random subspace of the feature~space.
Another ensemble learning technique is \textit{bagging}, which is training the
trees on random subsets of the training~set. In the case of Random Decision
Forests, the classifiers are decision trees. A decision tree is a tree where
each inner node uses one or more features to decide in which branch to descend.
Each leaf is a class.

One strength of Random Decision Forests compared to many other classifiers like
\glspl{SVM} and neural networks is that the scale of measure of the features
(nominal, ordinal, interval, ratio) can be arbitrary. Another advantage of
Random Decision Forests compared to \glspl{SVM}, for example, is the speed
of training and classification.

Random decision trees were extensively studied in the past 20~years and a
multitude of training algorithms has been proposed (e.g. ID3
in~\cite{quinlan1986induction}, C4.5 in~\cite{quinlan2014c4}). Possible
training hyperparameters are the measure to evaluate the \enquote{goodness of
split}~\cite{raey89empirical}, the number of decision trees being used, and if
the depth of the trees is restricted. Typically in the context of
classification, random decision trees are trained by adding new nodes until
each leaf contains only nodes of a single class or until it is not possible to
split further. This is called a \textit{stopping criterion}.

There are two typical training modes: \textit{Central axis projection} and
\textit{perceptron training}. In training, for each node a hyperplane is
searched which is optimal according to an error function.

Random Decision Forests with texton features (see \cref{subsubsec:textons}) are
applied in~\cite{shotton2008semantic} for segmentation. In the~\cite{MSCR-db}
dataset, they report a per-pixel accuracy rate of \SI{66.9}{\percent} for their
best system. This system needs \SI{415}{\milli\second} for the segmentation of
$\SI{320}{\pixel} \times \SI{213}{\pixel}$ images on a single
\SI{2.7}{\giga\hertz} core. On the Pascal VOC~2007 dataset, they report an
average per-pixel accuracy for their best segmentation system
of~\SI{42}{\percent}.

An excellent introduction to Random Decision Forests for semantic segmentation
is given by~\cite{schroff2008object}.

%!TEX root = vorlage.tex

\subsection{SVMs}\label{subsec:trad-SVM}%

\Glspl{SVM} are well-studied binary classifiers which can be described by five
central ideas. For those ideas, the training data is represented as
$(\mathbf{x}_i, y_i)$ where $\mathbf{x}_i$ is the feature vector and $y_i \in
\Set{-1, 1}$ the binary label for training example $i \in \Set{1, \dots, m}$.


\begin{enumerate}
    \item If data is linearly separable, it can be separated by a hyperplane.
          There is one hyperplane which maximizes the distance to the next
          datapoints (\textit{support vectors}). This hyperplane should be
          taken:\\
          \begin{equation*}
          \begin{aligned}
              \minimize_{\mathbf{w}, b}\,&\frac{1}{2} \|\mathbf{w}\|^2\\
              \text{s.t. }& \forall_{i=1}^m y_i \cdot \underbrace{(\langle \mathbf{w}, \mathbf{x}_i\rangle + b)}_{\mathclap{\sgn \text{ applied to this gives the classification}}} \geq 1
          \end{aligned}
          \end{equation*}
    \item Even if the underlying process which generates the features for the
          two classes is linearly separable, noise can make the data not
          separable. The introduction of \textit{slack variables} to relax the
          requirement of linear separability solves this problem. The trade-off
          between accepting some errors and a more complex model is weighted by
          a parameter $C \in \mathbb{R}_0^+$. The bigger $C$, the more errors
          are accepted. The new optimization problem is:
          \begin{equation*}
          \begin{aligned}
              \minimize_{\mathbf{w}}\,&\frac{1}{2} \|\mathbf{w}\|^2 + C \cdot \sum_{i=1}^m \xi_i\\
              \text{s.t. }& \forall_{i=1}^m y_i \cdot (\langle \mathbf{w}, \mathbf{x}_i\rangle + b) \geq 1 - \xi_i
          \end{aligned}
          \end{equation*}

          Note that \(0 \le \xi_i \le 1\) means that the data point is within
          the margin, whereas \(\xi_i \ge 1\) means it is misclassified. An
          \gls{SVM} with $C > 0$ is also called a \textit{soft-margin \gls{SVM}}.
    \item The primal problem is to find the normal vector $\mathbf{w}$ and the
          bias $b$. The dual problem is to express $\mathbf{w}$ as a linear
          combination of the training data $\mathbf{x}_i$:
          \[\mathbf{w} = \sum_{i=1}^m \alpha_i y_i \mathbf{x}_i\]
          where $y_i \in \Set{-1, 1}$ represents the class of the training
          example and $\alpha_i$ are Lagrange multipliers. The usage of
          Lagrange multipliers is explained with some examples
          in~\cite{smithlagrange}. The usage of the Lagrange multipliers
          $\alpha_i$ changes the optimization problem depend on the
          $\alpha_i$ which are weights for the feature vectors. It turns
          out that most $\alpha_i$ will be zero. The non-zero weighted vectors
          are called \textit{support vectors}.

          The optimization problem is now, according to~\cite{burges1998tutorial}:
          \begin{equation*}
          \begin{aligned}
              \maximize_{\mathbf{w}}\,& \sum_{i=1}^m \alpha_i - \frac{1}{2} \sum_{i=1}^m \sum_{j=1}^m \alpha_i \alpha_j y_i y_j \langle \mathbf{x}_i, \mathbf{x}_j \rangle\\
              \text{s.t. } & \forall_{i=1}^m 0 \leq \alpha_i \leq C\\
              \text{s.t. } & \sum_{i=1}^m \alpha_i y_i = 0
          \end{aligned}
          \end{equation*}
    \item Not every dataset is linearly separable. This problem is approached
          by transforming the feature vectors $\mathbf{x}$ with a non-linear
          mapping $\Phi$ into a higher dimensional (probably
          $\infty$-dimensional) space. As the feature vectors $\mathbf{x}$
          are only used within scalar product
          $\langle \mathbf{x}_i, \mathbf{x}_j \rangle$, it is not necessary to
          do the transformation. It is enough to do the calculation
          \[K(\mathbf{x}_i, \mathbf{x}_j) = \langle \mathbf{x}_i, \mathbf{x}_j \rangle\]

          This function $K$ is called a \textit{kernel}. The idea of never
          explicitly transforming the vectors $\mathbf{x}_i$ to the higher
          dimensional space is called the \textit{kernel trick}. Common kernels
          include the polynomial kernel
          \[K_P(\mathbf{x}_i, \mathbf{x}_j) = (\langle \mathbf{x}_i, \mathbf{x}_j \rangle + r)^p\]
          of degree $p$ and coefficient $r$, the Gaussian \gls{RBF} kernel
          \[K_{\text{Gauss}}(\mathbf{x}_i, \mathbf{x}_j) = e^{\frac{-\gamma\|\mathbf{x}_i - \mathbf{x}_j\|^2}{2 \sigma^2}}\]
          and the sigmoid kernel
          \[K_{\text{tanh}}(\mathbf{x}_i, \mathbf{x}_j) = \tanh(\gamma \langle \mathbf{x}_i, \mathbf{x}_j \rangle - r)\]
          where the parameter $\gamma$ determines how much influence single
          training examples have.
    \item The described \glspl{SVM} can only distinguish between two classes.
          Common strategies to expand those binary classifiers to multi-class
          classification is the \textit{one-vs-all} and the \textit{one-vs-one}
          strategy. In the one-vs-all strategy $n$ classifiers have to be
          trained which can distinguish one of the $n$ classes against all
          other classes. In the one-vs-one strategy $\frac{n^2 - n}{2}$
          classifiers are trained; one classifier for each pair of classes.
\end{enumerate}

A detailed description of \glspl{SVM} can be found in~\cite{burges1998tutorial}.

\Glspl{SVM} are used by \cite{yang2012layered} on the 2009 and 2010 PASCAL
segmentation challenge~\cite{everingham2010pascal}. They did not hand their
classifier in to the challenge itself, but calculated an average rank~of~7
among the different categories.

\cite{felzenszwalb2010object} also used an \gls{SVM} based method with \gls{HOG}
features and achieved the \nth{7}~rank in the 2010 PASCAL segmentation
challenge by mean accuracy. It needs about \SI{2}{\second} on a
\SI{2.8}{\giga\hertz} 8-core Intel processor.

%!TEX root = vorlage.tex

\subsection{Markov Random Fields}\label{subsec:markov-random-fields}
\Glspl{MRF} are undirected probabilistic graphical models which are wide-spread
model in computer vision. The overall idea of \glspl{MRF} is to assign a random
variable for each feature and a random variable for each pixel which gets
labeled. For example, a \gls{MRF} which is trained on images of the size
$\SI{224}{\pixel} \times \SI{224}{pixel}$ and gets the raw RGB values as
features has
\[\underbrace{224 \cdot 224 \cdot 3}_{\text{input}} + \underbrace{224 \cdot 224}_{\text{output}} = \num{200704}\]
random variables. Those random variables are conditionally independent, given
their local neighborhood. These (in)dependencies can be expressed with a graph.

Let $G=(\mathcal{V}, \mathcal{E})$ be the associated undirected graph of an
\gls{MRF} and $\mathcal{C}$ be the set of all maximal cliques in that graph.
Nodes represent random variables $\mathbf{x}, \mathbf{y}$ and edges represent
conditional dependencies. Just like in
\crefname{subsec:graph-based-image-segmentation}, the
4-neighborhood~\cite{shotton2006textonboost} and the 8-neighborhood are
reasonable choices for constructing the graph.

Typically, random variables $\mathbf{y}$ represent the class of a single pixel,
random variables $\mathbf{x}$ represent a pixel values and edges represent
pixel neighborhood in computer vision problems segmentation problems where
\glspl{MRF} are used. Accordingly, the random variables $\mathbf{y}$ live on
$1, \dots, \text{nr of classes}$ and the random variables $\mathbf{x}$
typically live on $0, \dots, 255$ or $[0, 1]$.

The probability of $\mathbf{x}, \mathbf{y}$ can be expressed as
\[P(\mathbf{x}, \mathbf{y}) = \frac{1}{Z} e^{-E(\mathbf{x}, \mathbf{y})}\]
where $Z = \sum_{\mathbf{x}, \mathbf{y}} e^{-E(\mathbf{x}, \mathbf{y})}$ is a normalization term called
the \textit{partition function} and $E$ is called the \textit{energy function}.
A common choice for the energy function is
\[E(\mathbf{x}, \mathbf{y}) = \sum_{c \in \mathcal{C}} \psi_c(\mathbf{x}, \mathbf{y})\]
where $\psi$ is called a \textit{clique potential}. One choice for cliques
of size two $\mathbf{x}, \mathbf{y} = (x_1, x_2)$ is~\cite{kato2006markov}
\[\psi_c(x_1, x_2) = w \delta(x_1, x_2) = \begin{cases}+w &\text{if } x_1 \neq x_2\\-w &\text{if } x_1 = x_2\end{cases}\]

According to~\cite{murphy2012machine}, the most common way of inference over
the posterior \gls{MRF} in computer vision problems is \gls{MAP} estimation.


Detailed introductions to \glspl{MRF} are given by
\cite{blake2011markov,murphy2012machine}. \glspl{MRF} are used by \cite{zhang2001segmentation} and \cite{moser2012markov}
for image segmentation.

% Characterizations of MRF:
% Label space: binary vs multi-label; homogeneous vs heterogeneous
% Order: unary vs pairwise vs higher-order
% Structure: chain vs tree vs grid vs general graph; neighborhood size
% Potentials: submodular, convex, compressible

% Markov Random Fields for Computer Vision (Part 1)
% http://users.cecs.anu.edu.au/~sgould/papers/part1-MLSS-2011.pdf  --- very nice!
% http://users.cecs.anu.edu.au/~sgould/papers/part2-MLSS-2011.pdf
% http://users.cecs.anu.edu.au/~sgould/papers/part3-MLSS-2011.pdf

% Markov Random Field Image Models and Their Applications to Computer Vision
% http://www.mathunion.org/ICM/ICM1986.2/Main/icm1986.2.1496.1517.ocr.pdf


\subsection{Conditional Random Fields}\label{subsec:conditional-random-fields}

\Glspl{CRF} are \glspl{MRF} where all clique potentials are conditioned on
input features~\cite{murphy2012machine}. This means, instead of learning the
distribution $P(\mathbf{y}, \mathbf{x})$, the task is reformulated to learn the
distribution $P(\mathbf{y}| \mathbf{x})$. One consequence of this reformulation
is that \glspl{CRF} need much less parameters as the distribution of
$\mathbf{x}$ does not have to be estimated. Another advantage of \glspl{CRF}
compared to \glspl{MRF} is that no distribution assumption about $\mathbf{x}$
has to be made.

A \gls{CRF} has the partition function $Z$:
\[Z(\mathbf{x}) = \sum_{\mathbf{y}} P(\mathbf{x}, \mathbf{y})\]

and joint probability distribution

\[P(\mathbf{y} | \mathbf{x}) = \frac{1}{Z(\mathbf{x})} \prod_{c \in \mathcal{C}} \psi_c(\mathbf{y}_c | \mathbf{x})\]

The simplest way to define the clique potentials $\psi$ is the count of the
class $\mathbf{y}_c$ given $\mathbf{x}$ added with a positive smoothing
constant to prevent the complete term from getting zero.

\Glspl{CRF} as described in~\cite{associative09} have reached top performance
in PASCAL VOC 2010~\cite{VOC2010Results} and are also used in
\cite{multiscale04,shotton2006textonboost} for semantic segmentation.

A method similar to \glspl{CRF} was proposed in~\cite{gonfaus2010harmony}.
The system of Gonfaus~et.al. ranked~\nth{1} by mean accuracy in the segmentation
task of the PASCAL VOC 2010 challenge~\cite{everingham2010pascal}.

An introduction to \glspl{CRF} is given by~\cite{sutton2011introduction}.

%!TEX root = vorlage.tex

\subsection{Post-processing methods}\label{subsec:post-processing-methods}%
Post-processing refine a found segmentation and remove obvious
errors. For example, the morphological operations \textit{opening} and
\textit{closing} can remove noise. The opening operation is a dilation followed
by a erosion. This removes tiny segments. The closing operation is a erosion
followed by a dilation. This removes tiny gaps in otherwise filled regions.
They were used in~\cite{chen1998image} for biomedical image segmentation.

Another way of refinement of the found segmentation is by adjusting the
segmentation to match close edges. This was used in~\cite{brox2011object} with
an ultra-metric contour map~\cite{arbelaez2009contours}.

Active contour models are another example of a post-processing
method~\cite{kass1988snakes}.


% TODO:
% \subsection{Discussion}\label{subsec:traditional-approaches-discussion}%
% According to \cite{pantofaru2005comparison}, the mean shift algorithm produces
% segmentations that correspond well to human perception, but it is sensitive to
% its parameters. Depending on them, different granularities of the segmentation
% can be achieved. \cite{pantofaru2005comparison} draws the conclusion, that
% the segmentations found by the graph based image segmentation approach
% in \cite{felzenszwalb2004efficient} are inferior to the segmentations found
% by the mean shift algorithm described in \cite{comaniciu2002mean}.
