%!TEX root = vorlage.tex

\section{Introduction}\label{sec:introduction}
Semantic segmentation is the task of clustering parts of images together which
belong to the same object class. This type of algorithm has several use-cases
such as detecting road signs~\cite{4220659}, detecting
tumors~\cite{moon2002automatic}, detecting medical instruments in
operations~\cite{wei1997automatic}, colon crypts
segmentation~\cite{cohen2015memory}, land use and land cover
classification~\cite{huang2002assessment}. In contrast, non-semantic
segmentation only clusters pixels together based on general characteristics of
single objects. Hence the task of non-semantic segmentation is not
well-defined, as many different segmentations might be acceptable.

Several applications of segmentation in medicine are listed
in~\cite{annurev.bioeng.2.1.315}.

Object detection, in comparison to semantic segmentation, has to distinguish
different instances of the same object. While having a semantic segmentation is
certainly a big advantage when trying to get object instances, there are a
couple of problems: neighboring pixels of the same class might belong to
different object instances and regions which are not connected my belong to the
same object instance. For example, a tree in front of a car which visually
divides the car into two parts.

This paper is organized as follows: It begins by giving a taxonomy of
segmentation algorithms in \cref{sec:taxonomy}. A summary of quality measures
and datasets which are used for semantic segmentation follows in
\cref{sec:evaluation-and-datasets}. A summary of traditional
segmentation algorithms and their characteristics follows in
\cref{sec:traditional-approaches}, as well as a brief, non-exhaustive
summary of recently published semantic segmentation algorithms which are based
on neural networks in \cref{sec:nn}. Finally, \cref{sec:problems} informs the
reader about typical problematic cases for segmentation algorithms.
