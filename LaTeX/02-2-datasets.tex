%!TEX root = vorlage.tex

\subsection{Datasets}

The computer vision community produced a couple of different datasets which are
publicly available. In the following, only the most widely used ones as well as
three medical databases are described. An overview over the quantity and
the kind of data is given by
\cref{table:segmentation-databases}.


\subsubsection{PASCAL VOC}

The PASCAL\footnote{\textbf{p}attern \textbf{a}nalysis, \textbf{s}tatistical
modelling and \textbf{c}omput\textbf{a}tional \textbf{l}earning, an EU network
of excellence} VOC\footnote{\textbf{V}isual \textbf{O}bject \textbf{C}lasses}
challenge was organized eight times with different datasets: Once every year
from 2005 to 2012~\cite{pascal-voc-2012}. Beginning with~2007, a segmentation
challenge was added~\cite{pascal-voc-2007}.

The dataset consists of annotated photographs from www.flicker.com, a photo
sharing website. There are multiple challenges for PASCAL VOC\@. The 2012
competition had five~challenges of which one is a segmentation challenge where
a single class label was given for each pixel. The classes are: aeroplane,
bicycle, bird, boat, bottle, bus, car, cat, chair, cow, dining table, dog,
horse, motorbike, person, potted plant, sheep, sofa, train, tv/monitor.

Although no new competitions will be held, new algorithms can be evaluated on
the 2010, 2011 and 2012 data via
\href{http://host.robots.ox.ac.uk:8080/}{http://host.robots.ox.ac.uk:8080/}

The PASCAL VOC segmentation challenges use the \textit{segmentation over union}
criterion (see \cref{subsec:quality-measures}).


\subsubsection{MSRCv2}\label{subsubsec:MSRCv2}

Microsoft Research has published a database of 591~photographs with pixel-level
annotation of 21~classes: aeroplane, bike, bird, boat, body, book, building,
car, cat, chair, cow, dog, face, flower, grass, road, sheep, sign, sky, tree,
water. Additionally, there is a \texttt{void} label for pixels which do not
belong to any of the 21~classes or which are close to the segmentation
boundary. This allows a \enquote{rough and quick hand-segmentation which does
not align exactly with the object boundaries}~\cite{shotton2006textonboost}.

\subsubsection{Medical Databases}

The Warwick-QU Dataset consists of 165~images with pixel-level annotation of
5~classes: \enquote{healthy, adenomatous, moderately differentiated,
moderately-to-poorly differentiated, and poorly
differentiated}~\cite{coelho2009nuclear}. This dataset is part of the
Gland Segmentation (GlaS) challenge.

The DIARETDB1~\cite{kalesnykiene2014diaretdb1} is a dataset of 89~images fundus
images. Those images show the interior surface of the eye. Fundus images can
be used to detect diabetic retinopathy. The images have four classes of coarse
annotations: hard and soft exudates, hemorrhages and red small dots.

20~test and additionally 20~training retinal fundus images are available
through the DRIVE data set~\cite{staal2004ridge}. The vessels were annotated.
Additionally, \cite{azzopardi2011detection} added vascular features.

The Open-CAS Endoscopic Datasets~\cite{maier2014can} are 60~images taken from
laparoscopic adrenalectomies and 60~images taken from laparoscopic pancreatic
resections. Those are from 3 surgical procedures each. Half of the data was
annotated by a medical expert for \enquote{medial instrument} and \enquote{no
medical instrument}. All images were labeled by anonymous untrained workers to
which they refer to as \textit{knowledge workers} (KWs). One crowd annotation
was obtained for each image by a majority vote on a pixel basis of
10~segmentations given by 10~different KWs.