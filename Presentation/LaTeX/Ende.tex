%!TEX root = Presentation-Thoma.tex
\section{Ende}
\subsection{Danke!}
\begin{frame}{Danke!}
    \begin{center}
        \Huge
	    Gibt es Fragen?
    \end{center}
\end{frame}

\subsection{Bildquellen}
\begin{frame}{Bildquellen}
\begin{itemize}
	\item J. M. Sharif, M. F. Miswan, M. A. Ngadi, Md Sah Hj Salam.
          \textit{Red Blood Cell Segmentation Using Masking and Watershed Algorithm: A Preliminary Study}. 2012.
    \item S. Hu, E. Hoffman, J. Reinhardt. \textit{Automatic lung segmentation for accurate quantitation of volumetric X-ray CT images}. 2001.
    \item D. L. Pham, C. Xu and J. L. Prince. \textit{A survey of Current Methods in Medical Image Segmentation}. 2000.
\end{itemize}
\end{frame}

\subsection{Literatur}
\begin{frame}{Literatur}
\begin{itemize}
    \item J. Shotton, J. Winn, C. Rother and  A. Criminisi: \textit{Textonboost: Joint appearance, shape and context modeling for multi-class object recognition and segmentation}. 2006.
    \item J. Shotton, M. Johnson and R. Cipolla: \textit{Semantic texton forests for image categorization and segmentation}. 2008.
    \item Y. Yang, S. Hallman, D. Ramanan and C. Fowlkes: \textit{Layered object models for image segmentation}. 2012.
    \item Insgesamt 119 Quellen, vgl. Paper für den Rest.
\end{itemize}
\end{frame}

\subsection{Folien, \LaTeX und Material}
\begin{frame}{Folien, \LaTeX und Material}
Der Foliensatz sowie die \LaTeX und Ti\textit{k}Z-Quellen sind unter

\href{https://github.com/MartinThoma/seminar-pixel-exact-classification}{github.com/MartinThoma/seminar-pixel-exact-classification}
\\

Kurz-URL:
\href{http://tinyurl.com/semantic-segmentation}{tinyurl.com/semantic-segmentation}
\end{frame}
